\documentclass[12pt]{article} 
\usepackage[italian]{babel}
\usepackage[utf8]{inputenc} 
\usepackage{geometry} 
\geometry{
  a4paper,
  top=2cm,
  bottom=2cm,
  outer=2cm,
  inner=3cm,
  bindingoffset=0cm
}
\usepackage{graphicx} 


%%% PACKAGES
\usepackage{paralist} % very flexible & customisable lists (eg. enumerate/itemize, etc.)
\usepackage{verbatim} % adds environment for commenting out blocks of text & for better verbatim
\usepackage{subfig} % make it possible to include more than one captioned figure/table in a single float

%%% HEADERS & FOOTERS
\usepackage{fancyhdr} % This should be set AFTER setting up the page geometry
\pagestyle{fancy} % options: empty , plain , fancy
\renewcommand{\headrulewidth}{0pt} % customise the layout...
\lhead{}\chead{}\rhead{}
\lfoot{}\cfoot{\thepage}\rfoot{}

%%% SECTION TITLE APPEARANCE
\usepackage{sectsty}
\allsectionsfont{\rmfamily\mdseries\upshape} % (See the fntguide.pdf for font help)
% (This matches ConTeXt defaults)

\usepackage{titlesec}

% Titoli in grassetto (solo il titolo, non il contenuto)
\titleformat{\section}{\normalfont\bfseries\Large}{\thesection}{1em}{}
\titleformat{\subsection}{\normalfont\bfseries\large}{\thesubsection}{1em}{}
\titleformat{\subsubsection}{\normalfont\bfseries\normalsize}{\thesubsubsection}{1em}{}

%%% ToC (table of contents) APPEARANCE
\usepackage[nottoc,notlof,notlot]{tocbibind} % Put the bibliography in the ToC
\usepackage[titles,subfigure]{tocloft} % Alter the style of the Table of Contents
\renewcommand{\cftsecfont}{\rmfamily\mdseries\upshape}
\renewcommand{\cftsecpagefont}{\rmfamily\mdseries\upshape} % No bold!

\usepackage[hidelinks]{hyperref}

\usepackage{newtxtext} % per il testo
\usepackage{newtxmath} % per formule matematiche abbinate
\usepackage{setspace}
\onehalfspacing

%%% END Article customizations

\begin{document}

\begin{center}
\subsection*{Sommario}
\end{center}
Il progetto di tirocinio presentato in questa tesi consiste nello sviluppo di un’applicazione mobile multipiattaforma tramite Flutter, un framework moderno e altamente performante per la creazione di interfacce utente native.
\\\\ L'obiettivo è la realizzazione di una piattaforma SaaS (Software as a Service) per il monitoraggio di ambienti di produzione digitali, in particolare servizi backend e API. Ogni utente ha la possibilità di configurare dei "monitor", associati a uno o più progetti, per controllare lo stato e l'affidabilità delle risorse attraverso le risposte HTTP (codici 2xx, 4xx, 5xx, ecc.).
\\\\L'applicazione permette di rilevare tempestivamente anomalie o interruzioni nei servizi, notificando in tempo reale eventuali errori e consentendo così una rapida risposta. Questo sistema si rivela particolarmente utile per agenzie e aziende che gestiscono infrastrutture digitali come siti web, database o microservizi.
\\\\Dal punto di vista commerciale, la piattaforma può essere monetizzata tramite un sistema di abbonamento mensile, con funzionalità scalabili in base al numero di progetti o monitor attivi. Questo approccio garantisce flessibilità sia per piccoli team sia per aziende strutturate, offrendo un valore concreto in termini di affidabilità e qualità del servizio.

\newpage

\tableofcontents

\newpage

\input{sezioni/0_introduzione.tex}
\newpage
\section{\textbf{Analisi preliminare}}
\newpage
\section{\textbf{Progettazione}}
\newpage
\section{\textbf{Descrizione implementazione/sviluppo delle varie parti}}
\newpage
\section{\textbf{Problematiche}}
\newpage
\section{\textbf{Risultati ottenuti}}
\newpage
\section{\textbf{Conclusioni e sviluppi futuri}}


\end{document}
