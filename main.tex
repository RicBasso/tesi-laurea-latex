\documentclass[12pt]{article} 
\usepackage[italian]{babel}
\usepackage[utf8]{inputenc} 
\usepackage{geometry} 
\geometry{a4paper} 
\usepackage{graphicx} 


%%% PACKAGES
\usepackage{paralist} % very flexible & customisable lists (eg. enumerate/itemize, etc.)
\usepackage{verbatim} % adds environment for commenting out blocks of text & for better verbatim
\usepackage{subfig} % make it possible to include more than one captioned figure/table in a single float

%%% HEADERS & FOOTERS
\usepackage{fancyhdr} % This should be set AFTER setting up the page geometry
\pagestyle{fancy} % options: empty , plain , fancy
\renewcommand{\headrulewidth}{0pt} % customise the layout...
\lhead{}\chead{}\rhead{}
\lfoot{}\cfoot{\thepage}\rfoot{}

%%% SECTION TITLE APPEARANCE
\usepackage{sectsty}
\allsectionsfont{\sffamily\mdseries\upshape} % (See the fntguide.pdf for font help)
% (This matches ConTeXt defaults)

%%% ToC (table of contents) APPEARANCE
\usepackage[nottoc,notlof,notlot]{tocbibind} % Put the bibliography in the ToC
\usepackage[titles,subfigure]{tocloft} % Alter the style of the Table of Contents
\renewcommand{\cftsecfont}{\rmfamily\mdseries\upshape}
\renewcommand{\cftsecpagefont}{\rmfamily\mdseries\upshape} % No bold!

%%% END Article customizations

\begin{document}

\begin{center}
\subsection*{Sommario}
\end{center}
Il progetto di tirocinio presentato in questa tesi consiste nello sviluppo di un’applicazione mobile multipiattaforma tramite Flutter, un framework moderno e altamente performante per la creazione di interfacce utente native.
\\\\ L'obiettivo è la realizzazione di una piattaforma SaaS (Software as a Service) per il monitoraggio di ambienti di produzione digitali, in particolare servizi backend e API. Ogni utente ha la possibilità di configurare dei "monitor", associati a uno o più progetti, per controllare lo stato e l'affidabilità delle risorse attraverso le risposte HTTP (codici 2xx, 4xx, 5xx, ecc.).
\\\\L'applicazione permette di rilevare tempestivamente anomalie o interruzioni nei servizi, notificando in tempo reale eventuali errori e consentendo così una rapida risposta. Questo sistema si rivela particolarmente utile per agenzie e aziende che gestiscono infrastrutture digitali come siti web, database o microservizi.
\\\\Dal punto di vista commerciale, la piattaforma può essere monetizzata tramite un sistema di abbonamento mensile, con funzionalità scalabili in base al numero di progetti o monitor attivi. Questo approccio garantisce flessibilità sia per piccoli team sia per aziende strutturate, offrendo un valore concreto in termini di affidabilità e qualità del servizio.


\newpage

\tableofcontents

\newpage



\section{Introduzione}

\subsection{Progettazione}

\subsubsection{Contesto e motivazione}
\subsubsection{Obiettivo}
\subsubsection{Stato dell’arte}
\paragraph{Vantaggi applicazione multi-platform}

\subsubsection{Descrizione UpApi}
\subsubsection{Scelte tecnologiche}
\paragraph{Flutter}
\paragraph{Node.js}
\paragraph{Cubit}

\subsubsection{Panoramica dei capitoli}
\paragraph{Obiettivi / specifiche / use case / requisiti}
\paragraph{Descrizione implementazione/sviluppo delle varie parti}
\paragraph{Problematiche}
\paragraph{Risultati ottenuti}
\paragraph{Conclusioni e sviluppi futuri}


\end{document}
